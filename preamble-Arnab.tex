\documentclass[11pt,a4paper,onesided,final, notitlepage]{article}
\usepackage[tmargin=2.5cm,bmargin=2.5cm,lmargin=3.7cm,rmargin=3.2cm]{geometry}
\usepackage[utf8]{inputenc}
\usepackage[english]{babel}
%\documentclass[psamsfonts]{amsart}

%----------Packages-----------------------------------------------------
\usepackage{amssymb,amsfonts}
\usepackage{amsthm}
%\usepackage{mathabx}%---For widecheck
\usepackage{calligra,mathrsfs}%---For mathscr
\usepackage{mathtools}%-------Loads amsmath
\usepackage{tikz-cd}%---------Commutative Diagrams, NOT loaded in tikz package, this is a special package easy to use better than amscd
\usepackage{tikz}%-------tikz package for diagrams and drawings
\usetikzlibrary{shapes.geometric, arrows}%---libraries loaded in tikz
\usepackage[backend=biber,style=alphabetic,sorting=nyt]{biblatex}%----biber(recommended) or bibtex(traditional) is the backbender that sorts the bib items, common styles are "numeric" and "alphabetic" and "draft", common sorting counter are "none" for no sorting
\usepackage{csquotes}%When using babel or polyglossia with biblatex, loading csquotes is recommended to ensure that quoted texts are typeset according to the rules of your main language.


%--------Theorem Environments--------------------------------------------
%theoremstyle{plain} --- default
\newtheorem{thm}{Theorem}[section]
\newtheorem{cor}[thm]{Corollary}
\newtheorem{prop}[thm]{Proposition}
\newtheorem{lem}[thm]{Lemma}
\newtheorem{conj}[thm]{Conjecture}
\newtheorem{quest}[thm]{Question}

\theoremstyle{definition}
\newtheorem{defn}[thm]{Definition}
\newtheorem{defns}[thm]{Definitions}
\newtheorem{con}[thm]{Construction}
\newtheorem{exmp}[thm]{Example}
\newtheorem{exmps}[thm]{Examples}
\newtheorem{notn}[thm]{Notation}
\newtheorem{notns}[thm]{Notations}
\newtheorem{addm}[thm]{Addendum}
\newtheorem{exer}[thm]{Exercise}

\theoremstyle{remark}
\newtheorem{com}[thm]{Comment}
\newtheorem{rem}[thm]{Remark}
\newtheorem{rems}[thm]{Remarks}
\newtheorem{warn}[thm]{Warning}
\newtheorem{sch}[thm]{Scholium}
%\makeatletter

%\let\c@equation\c@thm
%\def\els@aparagraph[#1]#2{\elsparagraph[#1]{#2\@addpunct{.}}}
%\def\els@bparagraph#1{\elsparagraph*{#1\@addpunct{.}}}
%\makeatother

\numberwithin{equation}{section}
%--------Environments define-------------------------------------------
\newenvironment{teqn}{\begin{equation}\begin{tikzcd}\displaystyle}{\end{tikzcd}\end{equation}}
\newenvironment{teqn*}{\begin{equation*}\begin{tikzcd}}{\end{tikzcd}\end{equation*}}
\newenvironment{tpic}{\begin{equation}\begin{tikzpicture}}{\end{tikzpicture}\end{equation}}
\newenvironment{tpic*}{\begin{equation*}\begin{tikzpicture}}{\end{tikzpicture}\end{equation*}}
%\renewenvironment{proof}{\paragraph{\textit{Proof.}}}{fill}

%--------Symbols define -----------------------------------------------
\DeclareMathOperator{\A}{\mathbb{A}}%-----mathbb
\DeclareMathOperator{\R}{\mathbb{R}}%-----mathbb
\DeclareMathOperator{\C}{\mathbb{C}}%-----mathbb
\DeclareMathOperator{\Q}{\mathbb{Q}}%-----mathbb
\DeclareMathOperator{\F}{\mathbb{F}} %------mathbb
\DeclareMathOperator{\Z}{\mathbb{Z}}%-----mathbb
\DeclareMathOperator{\p}{\mathbb{P}}
\DeclareMathOperator{\fp}{\mathfrak{p}}%--mathfrak
\DeclareMathOperator{\cO}{\mathcal{O}}
\DeclareMathOperator{\G}{\mathbb{G}}
\DeclareMathOperator{\derglob}{\mathit{H}}
\DeclareMathOperator{\cF}{\mathscr{F}}
\DeclareMathOperator{\cG}{\mathscr{G}}
\DeclareMathOperator{\cH}{\mathscr{H}}
\DeclareMathOperator{\cI}{\mathscr{I}}
\DeclareMathOperator{\cA}{\mathscr{A}}
\DeclareMathOperator{\cB}{\mathscr{B}}
\DeclareMathOperator{\cK}{\mathscr{K}}
\DeclareMathOperator{\cL}{\mathscr{L}}
\DeclareMathOperator{\cC}{\mathscr{C}}
\DeclareMathOperator{\cD}{\mathscr{D}}
\DeclareMathOperator{\cT}{\mathcal{T}}
\DeclareMathOperator{\fq}{\mathfrak{q}}%---mathcal
\DeclareMathOperator{\fa}{\mathfrak{a}}
\DeclareMathOperator{\fm}{\mathfrak{m}}
\DeclareMathOperator{\fX}{\mathfrak{X}}
\DeclareMathOperator{\cHom}{\mathscr{H}\text{\kern -3pt {\calligra\large om}}\,}
 %\renewcommand{\P}{\mathbb{P}}
 % \DeclareMathOperator {\nC}{\mathbb{C}^{\times}}
%               -------  Calculus   --------------


%           ------- Linear Algebra ---------------
\DeclareMathOperator{\im}{im}
\DeclareMathOperator{\id}{id}
\DeclareMathOperator{\Hom}{Hom}
\DeclareMathOperator{\End}{End}
\DeclareMathOperator{\Aut}{Aut}
\DeclareMathOperator{\rank}{rank}
\DeclareMathOperator{\Ass}{Ass}
\DeclareMathOperator{\Fin}{Fin}
\DeclareMathOperator{\Ch}{Ch}
\DeclareMathOperator{\Sets}{Sets}
\DeclareMathOperator{\fsets}{FSets}
\DeclareMathOperator{\Fr}{Fr}
\DeclareMathOperator{\GL}{GL}
\DeclareMathOperator{\PGL}{PGL}
\DeclareMathOperator{\SL}{SL}
% ---------------Algebra
\DeclareMathOperator{\gl}{gl\_dim}
\DeclareMathOperator{\tr}{trace}
\DeclareMathOperator{\Tor}{Tor}
\DeclareMathOperator{\Ext}{Ext}
\DeclareMathOperator{\gr}{gr}
\DeclareMathOperator{\Mod}{Mod}
\DeclareMathOperator{\Alg}{Alg}
\DeclareMathOperator{\height}{height}
\DeclareMathOperator{\depth}{depth}
\DeclareMathOperator{\codim}{codim}
\DeclareMathOperator{\coker}{coker}
\DeclareMathOperator{\ann}{ann}
\DeclareMathOperator{\pd}{\text{pd}}
\DeclareMathOperator{\Fitt}{Fitt}
\DeclareMathOperator{\Gal}{Gal}
\DeclareMathOperator{\op}{op}
%           -------- Geometry ---------------
\DeclareMathOperator{\spec}{Spec}
\DeclareMathOperator{\Sch}{Sch}
\DeclareMathOperator{\proj}{Proj}
\DeclareMathOperator{\Pic}{Pic} 
\DeclareMathOperator{\pico}{Pic^{0}}
\DeclareMathOperator{\Div}{Div}
\DeclareMathOperator{\ord}{ord}
\DeclareMathOperator{\obj}{Obj}
\DeclareMathOperator{\mor}{Mor}
\DeclareMathOperator{\Ind}{Index}
\DeclareMathOperator{\length}{length}
\DeclareMathOperator{\ic}{i.c.}
\DeclareMathOperator{\an}{an}

%          ---------- Arithmetic -------------
\DeclareMathOperator{\spa}{Spa}
\DeclareMathOperator{\perf}{Perf}
\DeclareMathOperator{\zar}{zar}
\DeclareMathOperator{\shv}{Sh}
\DeclareMathOperator{\Et}{Et}
\DeclareMathOperator{\Zar}{Zar}
\DeclareMathOperator{\et}{\Acute{e}t}
\DeclareMathOperator{\fet}{f\Acute{e}t}
\DeclareMathOperator{\afet}{af\Acute{e}t}
\DeclareMathOperator{\gal}{Gal}
\DeclareMathOperator{\galk}{\gal(\bar{K}/K)}
\DeclareMathOperator{\Char}{char}
\DeclareMathOperator{\fppf}{\textit{fppf}}
\DeclareMathOperator{\fpqc}{\textit{fpqc}}
%\DeclareMathOperator{\Qproots}{\mathbb{Q}_\text{p}\left(\text{p}^{1/\text{p}^{\infty}}\right)}
%\DeclareMathOperator{\eq}{\xlongrightarrow{\text{eq.}}}

%----------------------Commands define--------------------------------
\newcommand{\eq}[2][\text{eq.}]{\xrightarrow[#2]{#1}}
\newcommand{\tilt}[1]{{#1^{\flat}}}
\newcommand{\untilt}[1]{{#1^{\#}}}
\newcommand{\tight}[1]{{#1^{\ast}}}
\newcommand{\units}[1]{{#1^{\times}}}
\newcommand{\sep}[1]{{#1^{\text{sep}}}}
\newcommand{\br}[1]{\left(#1\right)}
\newcommand{\inv}[1]{\left[\frac{1}{#1}\right]}
\newcommand{\colim}[1]{\underset{#1}{\text{colim}}}
\newcommand{\twocolim}[1]{\underset{#1}{\text{2-colim}}}
\newcommand {\sub}{\mbox{SB}}
\newcommand{\cc}[1]{\mathcal{#1}}
\newcommand{\zero}[1]{{#1^0}}
\newcommand{\doublezero}[1]{{{#1}^{00}}}
\newcommand{\zz}[1]{#1^{00}}
\newcommand{\plus}[1]{#1^+}
\newcommand{\minus}[1]{#1^{-}}
\newcommand{\bounded}[1]{#1^{b}}
\newcommand{\ff}[1]{#1^{\flat}}
\newcommand{\Qproots}{\mathbb{Q}_p\left(p^{1/p^{\infty}}\right)}
\newcommand{\formal}[1]{[\![#1]\!]}
\newcommand{\parx}[2]{\frac{\partial #2}{\partial x_{#1}}}
\newcommand{\proots}[1]{{#1}^{1/p^{\infty}}}
\newcommand{\nroot}[2]{#1^{\frac{1}{#2}}}
\newcommand{\rationalnbd}[2]{\left\langle\frac{#1}{#2}\right\rangle}
\newcommand{\rationalmultnbd}[2]{\left\langle\frac{#1_1,\cdots ,#1_n}{#2}\right\rangle}
\newcommand{\qpowers}[2]{#1^{[#2]}}
\newcommand{\almostisom}{\cong^a}
\newcommand{\almostequal}{=^a}
\newcommand{\complex}[1]{#1^{\cdot}}
\newcommand{\pushforward}[1]{{#1}_{\ast}}
\newcommand{\pullback}[1]{{{#1}^{\ast}}}
\newcommand{\lowershriek}[1]{#1_{!}}
\newcommand{\uppershriek}[1]{#1^{!}}
\newcommand{\rightderived}[1]{R #1}
\newcommand{\leftderived}[1]{L #1}
\newcommand{\irightderived}[2]{R^{#1} #2}
\newcommand{\crightderived}[1]{R_c{#1}}
\newcommand{\ileftderived}[2]{L^{#1} #2}
\newcommand{\derived}[1]{\mathscr{D}(#1)}
\newcommand{\pderived}[1]{\mathscr{D}^{+}{#1}}
\newcommand{\mderived}[1]{\mathscr{D}^{-}{#1}}
\newcommand{\bderived}[1]{\mathscr{D}^{b}{#1}}
\newcommand{\dual}[1]{#1^{DUAL}}
%\numberwithin{eqution}{section}

%----------------------Latex3 Command define --------------------------
% \NewDocumentCommand{\foo}{o}{#1}
% \NewDocumentCommand{\formula}{s m m}{
%         \IfBooleanTF(#1){
%               \frac{#2_1,\cdots,#2_n}{#3}
%         }{
%                 \frac{#2}{#3}
%         }
        
% }
% \newcommand{\rationalopen}{}

%----------------------Tikz define-------------------------------------
\tikzstyle{textbox} = [rectangle, rounded corners, minimum width=3cm, minimum height=1cm,text width=3cm, text centered, draw=black, fill=none]
\tikzstyle{u} = [text centered, draw=none, fill=none]
\tikzstyle{arrow2} = [thick,->,>=stealth]
\tikzstyle{arrow1} = [U]
%-----------------------Misc.
\setcounter{tocdepth}{2}
